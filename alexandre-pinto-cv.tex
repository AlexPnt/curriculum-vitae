% (c) 2002 Matthew Boedicker <mboedick@mboedick.org> (original author) http://mboedick.org
% (c) 2003-2007 David J. Grant <davidgrant-at-gmail.com> http://www.davidgrant.ca
% (c) 2008-2014 Nathaniel Johnston <nathaniel@njohnston.ca> http://www.njohnston.ca
%
% Depending on your TeX distribution, you may need to download the revnum and longtable packages for this template to work!
%
%This work is licensed under the Creative Commons Attribution-Noncommercial-Share Alike 2.5 License. To view a copy of this license, visit http://creativecommons.org/licenses/by-nc-sa/2.5/ or send a letter to Creative Commons, 543 Howard Street, 5th Floor, San Francisco, California, 94105, USA.


%
% Edited by Alexandre Pinto
% 2017
%

\documentclass[usenames,dvipsnames,letterpaper,11pt]{article}
\newlength{\outerbordwidth}
\pagestyle{empty}
\raggedbottom
\raggedright
\usepackage{array}
\usepackage[svgnames]{xcolor}
\usepackage{enumerate}
\usepackage{framed}
\usepackage{longtable}
\usepackage{revnum}
\usepackage[pdftex,colorlinks=true,urlcolor=blue]{hyperref}
\usepackage{tocloft}
\usepackage[textwidth=18cm,textheight=25cm]{geometry} % Edit as required
\usepackage[utf8]{inputenc}
\usepackage{marvosym}
\usepackage[document]{ragged2e}
% cv-custom.tex
%
% Custom macro commands and packages
% for formatting a Curriculum Vitae.
%
% Author: Matthew Earnshaw <matt@earnshaw.org.uk>
% Inspired by http://www.cv-templates.info/2009/03/professional-cv-latex/

\pagestyle{empty}

% Packages
\usepackage[]{color} % For custom colours
\usepackage{titlesec} % For custom section headings
\usepackage{mdwlist} % For compact lists
\usepackage{marvosym} % For icons

% Hyperlink colour and style
\definecolor{linkcolour}{rgb}{0,0.2,0.6}
\hypersetup{colorlinks,breaklinks, urlcolor=linkcolour, linkcolor=linkcolour}

% Custom colour
\definecolor{lgray}{gray}{0.4}

% Custom list bullet
\renewcommand{\labelitemi}{$\succ$}

% Header commands
\newcommand{\name}[1]{\LARGE\textbf{#1}}
\newcommand{\address}[1]{\color{lgray}{#1}}
\newcommand{\tel}[1]{\Large\Telefon~\small{#1}}
\newcommand{\email}[1]{\Large\Letter~\href{mailto:#1}{\small{#1}}}
\newcommand{\web}[2]{\Large\Mundus~\href{#1}{\small{#2}}}
	
% Section headings
\titleformat{\section}{\large\scshape\raggedright}{}{0em}{}[\titlerule]
\titlespacing{\section}{0pt}{0.6cm}{5pt}
% Note: Create an environment for sections ?

% Full width tables
\newenvironment{ftabular}[1]
{\begin{tabular*}{0.95\textwidth}{@{\extracolsep{\fill}}#1}}
{\end{tabular*}}
 
%-----------------------------------------------------------
%Edit these values as you see fit

\setlength{\outerbordwidth}{3pt}  % Width of border outside of title bars
\definecolor{shadecolor}{gray}{0.75}  % Outer background color of title bars (0 = black, 1 = white)
\definecolor{shadecolorB}{gray}{0.93}  % Inner background color of title bars


%-----------------------------------------------------------
%Margin setup

\setlength{\evensidemargin}{-0.25in}
\setlength{\headheight}{0in}
\setlength{\headsep}{0in}
\setlength{\oddsidemargin}{-0.25in}
\setlength{\paperheight}{11in}
\setlength{\paperwidth}{8.5in}
\setlength{\tabcolsep}{0in}
\setlength{\textheight}{9.5in}
\setlength{\textwidth}{7in}
\setlength{\topmargin}{-0.3in}
\setlength{\topskip}{0in}
\setlength{\voffset}{0.1in}
\setlength\LTleft{0.2in} % needed to make longtable full-width
\setlength\LTright{0.2in}

%-----------------------------------------------------------
%Custom commands
\newcommand{\resitem}[1]{\item #1 \vspace{-2pt}}
\newcommand{\resheading}[1]{\vspace{8pt}
  \parbox{\textwidth}{\setlength{\FrameSep}{\fboxsep}
    \begin{shaded}
\setlength{\fboxsep}{0pt}\framebox[\textwidth][l]{\setlength{\fboxsep}{4pt}\fcolorbox{shadecolorB}{shadecolorB}{\textbf{\sffamily{\mbox{~}\makebox[6.762in][l]{\large #1} \vphantom{p\^{E}}}}}}
    \end{shaded}
  }\vspace{-5pt}
}

\newcommand{\colorhref}[3][blue]{\href{#2}{\color{#1}{#3}}}%

% the next four commands allow for the \ressubheading environment to be 1, 2, 3, or 4 subrows, depending on which command you use. This is admittedly hack-ish, and should probably be replaced by a single more flexible command (with optional arguments) in the future
\newcommand{\ressubheading}[4]{
\begin{tabular*}{6.5in}[t]{l@{\cftdotfill{\cftsecdotsep}\extracolsep{\fill}}r}
		\textbf{#1} & #2 \\
		\textit{#3} & \textit{#4} \\
\end{tabular*}\vspace{-6pt}}
\newcommand{\ressubheadingb}[6]{
\begin{tabular*}{6.5in}[t]{l@{\cftdotfill{\cftsecdotsep}\extracolsep{\fill}}r}
		\textbf{#1} & #2 \\
		\textit{#3} & \textit{#4} \\
		\textit{#5} & \textit{#6} \\
\end{tabular*}\vspace{-6pt}}
\newcommand{\ressubheadingc}[8]{
\begin{tabular*}{6.5in}[t]{l@{\cftdotfill{\cftsecdotsep}\extracolsep{\fill}}r}
		\textbf{#1} & #2 \\
		\textit{#3} & \textit{#4} \\
		\textit{#5} & \textit{#6} \\
		\textit{#7} & \textit{#8} \\
\end{tabular*}\vspace{-6pt}}
\newcommand\foo[9]{%
    \def\tempb{#2}%
    \def\tempc{#3}%
    \def\tempd{#4}%
    \def\tempe{#5}%
    \def\tempf{#6}%
    \def\tempg{#7}%
    \def\temph{#8}%
    \def\tempi{#9}%
    \foocontinued
}
\newcommand\foocontinued[7]{%
    % Do whatever you want with your 9+7 arguments here.
}

\newcommand{\ressubheadingd}[1]{
	\def\argten{#1}%
	\ressubheadingdb
}
\newcommand{\ressubheadingdb}[9]{
\begin{tabular*}{6.5in}[t]{l@{\cftdotfill{\cftsecdotsep}\extracolsep{\fill}}r}
		\textbf{\argten} & #1 \\
		\textit{#2} & \textit{#3} \\
		\textit{#4} & \textit{#5} \\
		\textit{#6} & \textit{#7} \\
		\textit{#8} & \textit{#9} \\
\end{tabular*}\vspace{-6pt}}


%-----------------------------------------------------------


\begin{document}
{\large \begin{tabular*}{7in}{l@{\extracolsep{\fill}}r} 
\textbf{\LARGE Alexandre Pinto}\smallskip & \smallskip \\
Coimbra, Portugal & 


\email{alexandpinto@gmail.com}\\ 
& \web{https://alexpnt.github.io}{Personal Website}\\
& \web{https://github.com/alexpnt}{Github}\\
& \web{https://www.linkedin.com/in/alexandre-pinto}{Linkedin}\\
\end{tabular*}}
\\

%%%%%%%%%%%%%%%%%%%%%%%%%%%%%%
\resheading{Work Experience}
%%%%%%%%%%%%%%%%%%%%%%%%%%%%%%
%
\begin{ftabular}{p{4cm}|p{12cm}}
\textsc{\small{Nov 2017 - }} & \textbf{Machine Learning Engineer -} \textbf{\href{https://www.wit-software.com/}{WIT software}} \\

  
\end{ftabular}
\vspace{0.25cm}

\begin{ftabular}{p{4cm}|p{12cm}}
\textsc{\small{Nov 2016 - Oct 2017}} & \textbf{Backend Developer -} \textbf{\href{http://www.ubiwhere.com/en/}{Ubiwhere}} \\
 & I was part of the core team responsible for developing a smart water consumption platform for the city of Porto, Portugal. This meant building a system with high-availability, multi-tenancy and concurrency support. My job involved implementing REST APIs and developing the core of the backend web services. Furthermore, I was also involved in the development of the frontend mobile app, used by hundreds of users and available in multiple mobile operative systems such as Android and iOS.  \\
 & Technologies used: Django, Django REST Framework, Python Ecosystem, Ionic, Docker, GNU/Linux, git. \\
  
\end{ftabular}
\vspace{0.25cm}

\begin{ftabular}{p{4cm}|p{12cm}}
\textsc{\small{Oct 2015 - Jul 2016}} & \textbf{Research Intern -} \textbf{\href{https://www.inesctec.pt/ip-en?set_language=en\&cl=en}{INESCTEC}} / \textbf{\href{https://www.cisuc.uc.pt/}{CISUC}} \\
 & My job was to develop a filter system that classifies public social data according to their potential relevance to a general audience, filtering out irrelevant information and relying primarily on linguistic features, thereby confirming if relevance can be predicted from a set of journalistic criteria.\\
 & Technologies used: scikit-learn, NLTK, numPy, matplotlib, pandas, seaborn, PyQt5.\\
  
\end{ftabular}
\vspace{0.25cm}

\begin{ftabular}{p{4cm}|p{12cm}}
\textsc{\small{Oct 2013 - Mar 2014}} & \textbf{Software Developer -} \textbf{\href{http://ipn.pt/laboratorio/LIS}{Pedro Nunes Institute (IPN)}} \\
 & I maintained and developed new features for the information systems.\\
 & Technologies used: Java Struts, Ruby on Rails, Git.\\

\end{ftabular}


%%%%%%%%%%%%%%%%%%%%%%%%%%%%%%
\resheading{Relevant Academic Projects}
%%%%%%%%%%%%%%%%%%%%%%%%%%%%%%



\vspace{0.25cm}

\begin{ftabular}{p{4cm}|p{12cm}}
\textsc{\small{Mar 2016 - May 2016}\hspace{1cm}} & \textbf{Default Credit Card Prediction} \\
 & This project was carried out in the context of the Pattern Recognition course. The goal of this project was to develop classifiers to predict if a given client would be able to pay (or not) its credit card in the next month. The project followed the various steps of a typical machine learning pipeline (data preprocessing, feature selection/reduction, classification and evaluation). \\
 & Technologies used: scikit-learn, numPy, matplotlib, pandas, seaborn, PyQt5.\\
 & 
 
\end{ftabular}

\vspace{0.25cm}

\begin{ftabular}{p{4cm}|p{12cm}}
\textsc{\small{Feb 2014 - May 2014}\hspace{1cm}} & \textbf{Predicting the memorability of images} \\
 & This project was carried out in the context of the Artificial Intelligence course. The goal of this project was the automatic classification of images into a degree of memorability by computing their levels of attention according to a set of dimensions.\\
 & Technologies used: scikit-learn.\\
 & 
 
\end{ftabular}

\vspace{0.25cm}

\begin{ftabular}{p{4cm}|p{12cm}}
\textsc{\small{Feb 2014 - May 2014}\hspace{1cm}} & \textbf{Semantic Search and
Recommendation in eCommerce} \\
 & This project was carried out in the context of the Semantic Web course. The objective of the project was to build an ecommerce website where users could search for electronic products, browse product categories and get recommendations.\\
 & Technologies used: Protégé (Ontology editor), Apache Jena, Apache Tomcat.\\
 &
 
\end{ftabular}

\vspace{0.25cm}

\begin{ftabular}{p{4cm}|p{12cm}}
\textsc{\small{Feb 2014 - May 2014}\hspace{1cm}} & \textbf{Expert Contact} \\
 & This project was carried out in the context of the Software Project Management course. The objective of the project was to build a new way of communication between the nurses and patients with breast cancer during chemotherapy sessions. This project was a partnership between the research team working at Institute of Health and Care Sciences of the University of Gotemburg and the University of Coimbra and was conducted by a multidisciplinary team.\\
 & Technologies used: Struts2, Hibernate Generic D.A.O. Framework, Bootstrap, Git.\\
 & Role in the team: Developer.
 
\end{ftabular}
%%%%%%%%%%%%%%%%%%%%%%%%%%%%%%
\resheading{Education and training}
%%%%%%%%%%%%%%%%%%%%%%%%%%%%%%
\begin{itemize}
\item
	\ressubheading{University of Coimbra}{}{Master Degree in Informatics Engineering - Intelligent Systems}{2013 -- 2016}
	\begin{itemize}
		\resitem{Graduated with 15/20 average}
		\resitem{Dissertation titled “Classification of Social Media Posts according to their Relevance”}
	\end{itemize}
	\ressubheading{}{}{Relevant Courses:}{}
	\begin{itemize}
		\resitem{Pattern Recognition, Artificial Intelligence}
		\resitem{Evolutionary Computation, Adaptive Computation}
		\resitem{Semantic Web, Internet Applications}
		\resitem{Project Management, Systems Integration}
		\resitem{Information Theory, Statistics, Technical Communication}
	\end{itemize}
	
\item
	\ressubheading{University of Coimbra}{}{Bachelor Degree in Informatics Engineering}{2010 -- 2013}
	\begin{itemize}
		\resitem{Admission Grade: 17.55/20 }
		\resitem{Graduated with 16/20 average}
	\end{itemize}
\item
	\ressubheading{High School - Quinta das Flores}{}{Science and Technology Course }{2007 -- 2010}
	\begin{itemize}
		\resitem{Graduated with 16/20 average}
	\end{itemize}

\end{itemize}

\clearpage
%%%%%%%%%%%%%%%%%%%%%%%%%%%%%%
\resheading{Technical Skills}
%%%%%%%%%%%%%%%%%%%%%%%%%%%%%%
\begin{itemize*}
\item{\textbf{Programming Languages:} Proficient in Python, C and Java. Additional knowledge in ActionScript 3.0 and Matlab.}
\item{\textbf{Semantic Web:} Ontologies Representation (RDF,OWL), Triple Stores, SPARQL, Apache Jena, NLTK.}
\item{\textbf{Artificial Intelligence:} Evolutionary Computation, Supervised/Unsupervised Learning Algorithms, Machine Learning, NLP.}
\item{\textbf{Data Structures and Algorithms:} Knowledge of different Algorithmic Paradigms.}
\item{\textbf{Control Version Systems:} Git.}
\item{\textbf{Web Frameworks:} Struts2, Django, Django Rest Framework, AngularJS 1, Bottle, Flask, Falcon. Familiar with Rails.}
\item{\textbf{Hybrid Mobile App Frameworks:} Ionic.}
\item{\textbf{Machine Learning Tools:} Keras, Tensorflow, scikit-learn, SciPy stack (NumPy, Matplotlib, pandas), seaborn, Weka. }
\item{\textbf{Relational Databases:} PostgreSQL. Familiar with MySQL and Oracle.}
\item{\textbf{NoSQL Databases:} MongoDB.}
\item{\textbf{Development Tools:} PyCharm, IntelliJ, Datagrip, Eclipse, Netbeans, Sublime Text Editor.}
\item{\textbf{Deployment and Infrastructure:} Docker.}
\item{\textbf{Web:} Proficient with HTML. Familiar with CSS and Bootstrap.}
\item{\textbf{Operating Systems:} Competent in GNU/Linux and Windows.}
\item{\textbf{Productivity/Project Management Tools:} LaTeX, Trello, Slack.}
\item{\textbf{Languages:} Portuguese (fluent,native), English (Very Good).}
\item{\textbf{Professional:} Self-motivated, Self-learner, Team Player, Planning and Organizational Skills.}
\item{\textbf{Activities \& Interests:} Reading, Exercising, Programming by passion and hobby.}

\end{itemize*}

%%%%%%%%%%%%%%%%%%%%%%%%%%%%%%
\resheading{Publications}
%%%%%%%%%%%%%%%%%%%%%%%%%%%%%%
\begin{itemize}
\item
	\ressubheading{{\colorhref[black]{https://link.springer.com/article/10.1007\%2Fs00354-017-0015-1}{Predicting the Relevance of Social Media Posts Based on Linguistic Features}}}{}{Alexandre Pinto and Gonçalo Oliveira, H, and Alves, A., New Generation Computing, 2017}{}

	
\item
	\ressubheading{\colorhref[black]{http://drops.dagstuhl.de/opus/volltexte/2016/6008/pdf/OASIcs-SLATE-2016-3.pdf}{Comparing the Performance of Different NLP Toolkits in Formal and Social Media Text}}{}{Alexandre Pinto and Gonçalo Oliveira, H, and Alves, A., pp 1--16, vol 51, SLATE, 2016}{}

\end{itemize}

%%%%%%%%%%%%%%%%%%%%%%%%%%%%%%
\resheading{Awards, Grants \& Honours}
%%%%%%%%%%%%%%%%%%%%%%%%%%%%%%
	\vspace{-10pt}
	\begin{center}\begin{longtable}{l@{\extracolsep{\fill}}r}
		\multicolumn{2}{c}{Award to the 3\% Best Students \cftdotfill{\cftdotsep}2010 -- 2011}\\
		\multicolumn{2}{c}{Award to the 3\% Best Students \cftdotfill{\cftdotsep}2011 -- 2012}\\
		\vphantom{E}
\end{longtable}
\end{center}\vspace*{-50pt}


\end{document}